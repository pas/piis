% This line specifies the type of the document
\documentclass[10pt,a4paper]{scrartcl}
% possbile options are the fontsize (here "10pt")
%   the paper size (here "a4paper")
%   the pagetype (not present: "twopage" or "oneside")

% there exist various kind of documentclasses
%   article / scrticle
%   report / scrreport
%   book / scrbook
%   slides
%   letter / scrlttr2

%=============================================================================
% In this section follows the inclusion of different additional packages.
% Each package contains a certain set of additional features like special
% formatting language settings or for instance, tables.

% Make sure the input is handled correctly
\usepackage[utf8]{inputenc}

% This packages enables fancy text formatting / underlining ...
\usepackage[normalem]{ulem}

% Latex math support
% ftp://ftp.ams.org/pub/tex/doc/amsmath/short-math-guide.pdf
\usepackage{amsmath}
% Add special math symbols
\usepackage{amssymb}

% Enable citations of scientific documents
\usepackage{cite}

% Enable \includegraphics for easy insertion of graphics into a latex doc
\usepackage{graphicx}

% Enable adding several figures in a row (not used in this doc, yet)
\usepackage{subfigure}

% Add support for \xspace a small space
\usepackage{xspace}

%----------------------------------------------------------------------------
% This section sets the properties for listings. Listings are used to display
% verbatim (unformatted and with a fixed width font style) text like source
% code or terminal sessions.  Source code formatting can be set in detail:
\usepackage{color}
\usepackage{listings}
	% Global / primary settings for source code listing package
\definecolor{lightyellow}{rgb}{1,0.94,0.51}
\lstset{
    basicstyle=\ttfamily,
    showspaces=false,
    frame=single,
    xleftmargin=10pt,
    xrightmargin=10pt,
    framerule=0pt,
    aboveskip=10pt,
    belowskip=5pt,
    showstringspaces=false,
    captionpos=b, 
    columns=fullflexible,
    backgroundcolor=\color{lightyellow}}

% Define the listing shortcuts for terminal code (bash)
\lstnewenvironment{terminalcode}[1][]{\lstset{language=bash,#1}}{} 

%----------------------------------------------------------------------------

% The hyperref packages enables links in the outputed pdf enabled links in
% pdf, but paint them normal in black (changed it in blue).
\usepackage[pdftex=true, colorlinks=true, urlcolor=blue, 
			linkcolor=black,pagecolor=black,citecolor=black,
			bookmarks=true]{hyperref}

%=============================================================================

% Here the properties for the title of the document are set
\date{HS 2010 University Bern}
\author{PASCAL YVES ZAUGG}
\title{Trying to understand the tutorial}

%=============================================================================
% Past this line the real content of the file starts with \begin{document} and
% ends with \end{document} at the very end of this file
\begin{document}

% put the title page here
\maketitle

% put the table of contents here
\tableofcontents

% add a new page
\newpage
%=============================================================================
% starting a new section
\section{Documentation with Latex}
%=============================================================================
\subsection{Introduction} 

% \LaTeX displays the latex logo, \xpsace a tiny space
In this section we explain some \LaTeX\xspace details and different formatting
commands.

Whenever you need to lookup a certain symbol for \LaTeX\xspace we suggest you
to use the online recognition tool \texttt{detexify} at
\url{http://detexify.kirelabs.org/}.


%=============================================================================
\subsection{Common Commands}
% ----------------------------------------------------------------------------
\subsubsection{Sectioning}
Depening on the documentclass given in the very beginning of this file there
exist several sectioning levels:
% here we begin a list again but of another type named "enumerate":
\begin{enumerate}
\item{} \verb$\section{NAME}$
\item{} \verb$\subsection{NAME}$
\item{} \verb$\subsubsection{NAME}$
\item{} \verb$\paragraph{NAME}$
\end{enumerate}

% the \noindent causes the folowing block not to be indented ( = having a 
% certain space at the beginning)
\noindent To enforce \LaTeX to use a newline add a double slash \verb$\\$ at 
the end of a line.

% ----------------------------------------------------------------------------
\subsubsection{Font size and style}
% here we begin a new table having three columns "{lll}". Each "l" represents
% a column being left aligned. Other possible values would be "c" (centered)
% or "r" (right aligned):
\begin{tabular}{lll}
%                   each column is separated with an ampersand "&"
\verb$\rm$			& \rm{A normal text}\\ 
%                                         each new row is ended with "\\"
\verb$\sl$ 			& \sl{An italic text}\\
% \verb$$ does the same as \texttt{} but also allows commands to be displayed
% in verbatim
\verb$\bf$ 			& \bf{A bold text}\\
\verb$\tiny$ 		& \tiny{A tiny ext}\\
\verb$\scriptsize$ 	& \scriptsize{A very, very small text}\\
\verb$\footnotesize$& \footnotesize{A very small text}\\
\verb$\small$ 		& \small{A small text}\\
\verb$\large$ 		& \large{A big text}\\
\verb$\Large$ 		& \Large{A bigger text}\\
\verb$\LARGE$ 		& \LARGE{An even bigger text}\\
\verb$\huge$ 	    & \huge{A huge text}\\
\verb$\Huge$ 	    & \Huge{An enormous huge text}\\
\verb$\emph$ 	    & \emph{An emphasized text} \\
\verb$\underline$ 	& \underline{An underlined text} \uline{and here using the ulem-package}\\
\verb$\texttt$ 		& \texttt{function goto(int a) { ... } }\\
\verb$\uuline$ 		& \uuline{A double underlined text using the ulem-package} \\
\verb$\uwave$ 		& \uwave{A wavy underlined text using the ulem-package} \\
\verb$\sout$ 	    & \sout{A crossed trough text using the ulem-package}\\
\verb$\xout$ 	    & \xout{A deleted text using the ulem-package}\\
\end{tabular}

% ----------------------------------------------------------------------------
\subsubsection{Notes}
To create a footnote use the \verb$\footnote{YOUR NOTE}$
command\footnote{\dots as you can see here.}. \\ If you want to put a remark
at side of a page use \verb$\marginpar$.  \marginpar{This is a note at the
border of the page.}

% ----------------------------------------------------------------------------
\subsubsection{Lists}
There exist several list types in \LaTeX. You start a list by adding a
\verb$\being{LISTTYPE}$ and end it with an \verb$\end{LISTTYPE}$. A list item
is added with a \verb$\item$ between the \texttt{begin} and \texttt{end}.
\texttt{LISTTYPE} can be one of the following list:

\begin{itemize}
\item \texttt{enumerate}
\item \texttt{itemize}
\item \texttt{description} with \verb$\item[topic]$
\end{itemize}
% By adding a \noindet the next line is not indented ;)
\noindent Note that you can nest lists if you want to.

\begin{enumerate}
% start a new item named "e4"
\item{e4} 	
% add a list into the item named "e4"
	\begin{enumerate}
% add items to the nested list
	\item{e4}   e5
	\item Lc4 d6
% end the nested list
	\end{enumerate}
\item Lc4 d6
\end{enumerate}


% ----------------------------------------------------------------------------
\section{Vim}

\subsection{Generalities}
\begin{description}
\item{:q} Quit vim.
\item{:q!} Quit vim and ignore warnings.
\item{:w} Save file.
\item{:w filename.txt} Save file to filename.txt.
\item{:!} Start an extern command.
\end{description}

\subsection{Insert-Mode}
\begin{description}
\item{i} Start Insert-Mode in front of the cursor (\textbf{i}nsert).
\item{a} Start Insert-Mode behind the cursor (\textbf{a}ppend).
\item{A} Start Insert-Mode at the end of the line (\textbf{A}ppend).
\item{R} Start Replace-Mode (Existing characters are replaced while typing)
\item{r{\sl CHARACTER}} Replace character at cursor position with CHARACTER.
\end{description}

\subsection{Jumping between lines}
\begin{description}
\item{{\sl LINENUMBER}G} Cursor jumps to the line number previously typed. Example: 450G jumps to the line 450. NOTE: You won't see the line number while typing it.
\item{gg} Cursor jumps to the start of the file.
\item{G} Cursor jumps to the end of the file. 
\end{description}

\subsection{Deleting}
\begin{description}
\item{dd} Deleting a whole line.
\item{d\verb&$&} Deleting from cursor to the end of the line.
\item{dw} Deleting one word.
\end{description}

\noindent Some stuff which fits in here but doesn't use the d-command
\begin{description}
\item{x} Deleting and saving one character.
\item{{\sl NUMBER}x} Deleting and saving the declared count of characters.
\item{:j!} Deleting the newline at the end and \textbf{j}oining the next line.
\end{description}

\subsection{Search}
\begin{description}
\item{\verb$\KEYWORD$} Search KEYWORD in file.
\item{n} Go to the next match of KEYWORD.
\item{N} Go to previous match of KEYWORD.
\end{description}
\section{Bash-Commands}

\subsection{Generalities}
\begin {description}
\item{.} stands for the working directory.
\item{..} stands for parent directory of the working directory.
\item{\verb$~$} stands for home directory.
\end {description}
\subsection{cd - change directories}

With cd YOUR\verb$_$PREFERED\verb$_$DIRECTORY you can change your current directory. With cd .. you go one directory up. NOTE: Use tab-completion to make your live easier.

\subsection{ls - list directory contents}

With ls [ARGUMENT] YOUR\verb$_$PREFERED\verb$_$DIRECTORY you can show the content of a directory you prefered. Without a directory it lists the content of the current directory. Use the -l argument to list the content line by line and with some additional informations . Use the -R argument to list the content of subfolders as well. Use the -1 argument to list the content one file per line. 

\subsection{pwd - print name of current/working directory}

With pwd [ARGUMENT] you can see the path of your working directory. Use -p to avoid all symlinks.

\subsection{mkdir - make directories}

With mkdir [ARGUMENT] YOUR\verb$_$DIRECTORY you can make your own directories. Use -p to avoid errors and parent directories ar created as needed.

\subsection{touch - change file timestamps}

With touch [ARGUMENT] FILENAME you can change the file timestamp to the current time. If FILENAME does not exist it is created empty. Use -c if you want to be sure that there is no new file is created. Use --time=DATE change access date. Use -t DATETIME (format: CCYYMMDDhhmm) to change date and access time.

\subsection{mv - move (rename) directories}

With mv [ARGUMENT] SOURCE DESTINATION you can either rename a directory or move it. Use -i to prompt before overwriting. Use -u to update files (only newer files are moved). To move all files inside a directory use DIRECTORYNAME/* as SOURCE. To move only files with a use DIRECTORYNAME/a* as SOURCE.

\subsection{cp - copy files and directories}

With cp [ARGUMENT] SOURCE DESTINATION you can copy files and directories. Use -r to copy subfolders as well.

\subsection{rm - remove (unlink) files and directories}

With rm [ARGUMENT] FILENAME [ANOTHERFILENAME] you can remove directories and files. Use -r to remove a directory and all its subfolders recoursivly. You can remove more than one file in adding them at the end.

\subsection{cat - concatenate files and print on the standard output}

A file's content is directly printed to the terminal. Terminal commands are directly triggered. Use -v to give out a file but not triggering something.

\subsection{ssh - remote login program}

\subsection{grep - print lines matching a pattern}
With grep [ARGUMENT] PATTERN [FILENAME] you can print the lines matching PATTERN in a file. Use -c to count the lines where PATTERN occurs. Use -n to print linenumber in the file as well. Use -v to invert the result (all non-matching lines are printed). Use -i to ignore case distinctions.
\subsection{less - forward and backward movement through files}
With less [ARGUMENT] FILENAME you can scroll forward and backward through FILENAME. Use \verb$\PATTERN$ while running less to search for PATTERN. Use n to search next match. Use N to return to the previous match.

\subsection{sort - sort lines of text files}
With sort [ARGUMENT] FILENAME you can sort the lines of FILENAME. Use -n to sort by numbers. Use -r to sort in reversed order.

\subsection{wc -print newline, word, and byte counts for each file}
With wc [ARGUMENT] FILENAME you can count words, lines and characters and print the result. Use -w to count words. Use -l to count lines. Use -c to count characters.

\subsection{du - estimate file space usage}
With du [ARGUMENT] [DIRNAME] you can estimate file space usage. Use -h to make the output human-readable (summarizes Bytes to Megabytes etc.). Use -s to print only the summarized result. Use a to print file space for files as well. 

\subsection{find - search for files in a directory hierarchy}
With find [DIRNAME] [ARGUMENT] you can find directories and files. Use -type f to look only for files. Use -type d to look only for directories. Use -regex PATTERN to use a pattern to find files or directories. Use -iregex PATTERN to use the regex without case sensivity. Use .* in regex to tell find that there can be characters in front or at the end. 

\paragraph{Examples:} Some examples to understand the way find works.
\begin{terminalcode}
find -type f -regex .*r.*
\end{terminalcode}
Looks for all files containing an 'r'.

\begin{terminalcode}
find -type d -iregex .*r.*
\end{terminalcode}
Looks for all directories containing an 'r' or an 'R'.
%=============================================================================
\section{Ruby Programming}

Everything I learned about Ruby-Programming.

\subsection{Modules}

Modules are grouping similar methods together.

\subsection{Bash-Commands}

To use bash-command in ruby use \texttt{\`{}YOURBASHCOMMAND\`} (Backticks).

\subsection{Strings}

Strings are declared by \verb$#{name}$

\begin{description}
	\item{STRING.size} Returns the length of STRING.
	\item{STRING.upcase} Returns the STRING in uppercases.
	\item{STRING.downcase} Returns STRING in lowercases.
	\item{STRING.reverse} Returns STRING reversed.
	\item{STRING.chomp} Returns STRING without newline at the end (if there is one).
	\item{STRING.chop} Returns STRING without the last character.
\end{description}

\subsection{Arrays}
Creating an array:

\begin{terminalcode}
%w{the quick brown fox jumps over the lazy dog}
\end{terminalcode}

\begin{description}
\item{\texttt{ARRAY.size}} Find out how many items the list contains
\item{\texttt{ARRAY.sort}} Sort the array alphabetly
\item{\texttt{ARRAY.collect \{ |VARIABLENAME| block \}}} Invokes block ones for each element. Returns all values.
\item{\texttt{ARRAY.reject \{ |VARIABLENAME| block \}}} Invokes block ones for each element. Returns all false values.
\item{\texttt{ARRAY.select \{ |VARIABLENAME| block \}}} Invokes block ones for each element. Returns all true values.
\item{\texttt{ARRAY.each \{ |VARIABLENAME| block \}}} Invokes block ones for each element. Returns all elements of ARRAY.
\item{\texttt{ARRAY.join ('SEPERATING\_ELEMENT')  }} Joins all elements of ARRAY to one string seperated by SEPERARTING\_ELEMENT.
\item{\texttt{ARRAY.first}} Returns the first element of ARRAY.
\item{\texttt{ARRAY.last}} Returns the last element of ARRAY.
\item{\texttt{ARRAY+ARRAY2}} Returns a new list containing ARRAY and ARRAY2.
\item{\texttt{ARRAY*NUMBER}} Returns a new list containing the elements of ARRAY as often as NUMBER.
\end{description}

\subsection{Methods} 

Defining a method:

\begin{terminalcode}
def NAME
end
\end{terminalcode}

\noindent Defining a default variable:

\begin{terminalcode}
def h (name = "World")
  puts "Hello \verb$#{name}$"
end
\end{terminalcode}

\subsection{Objects}

Asking if a method responds (exists) for this type of object for this type of object

\begin{terminalcode}
OBJECT.respond.to?("METHODNAME")
\end{terminalcode}


%=============================================================================

\subsection{Control structures}

\paragraph{if}
When CONDITION is true then the block until end is executed. Else the program jumps to end.
\begin{terminalcode}
if CONDITION
  #add something    
end
\end{terminalcode}
\paragraph{if-else}
When CONDITION is true the block till else is executed. And then the programm jumps to end. Else the block from else till end is executed.
\begin{terminalcode}
if CONDITION
  #add something
else
  #add something else
end
\end{terminalcode}

\paragraph{while}
As long CONDITION returns true the block from while to end is executed endlessly.
\begin{terminalcode}
while CONDITION 
  #Change CONDITION here! 
end  
\end{terminalcode}

\paragraph{for}

Starting from NUMBER each time the program reaches 'end' the variable i is raised by one and starts again until i is bigger than NUMBER2.

\begin{terminalcode}
for i in (NUMBER..NUMBER2) do
  #do something here with i
end
\end{terminalcode}

\noindent NOTE: do is optional unless you write everything in one line.

\begin{terminalcode}
for i in (NUMBER..NUMBER2) do puts i end
\end{terminalcode}

\paragraph{case}

If VARIABLE  matches MATCH1 then case returns RESULT1. If VARIABLE  matches MATCH2 then case returns RESULT2. Else case returns RESULT3.
\begin{terminalcode}
case VARIABLE
  when MATCH1 then RESULT1
  when MATCH2 then RESULT2
  else RESULT3
end
\end{terminalcode}

\noindent A working example: 

\begin{terminalcode}
a=240
b = case a
      when 1..50 then Small
      when 51..100 then Medium
      when 101..150 then Large
      when 151..250 then X-Large
      when 201..250 then XX-Large
      else No such size exists
    end
    puts ``Your size is #{b}.''
\end{terminalcode}

\noindent Output: XX-Large

\subsection{Example}

The \#-sign declares a comment. Those lines are ignored by the compiler.
\begin{terminalcode}
# Script by Adrian Kuhn
\end{terminalcode}
Assigning the value "notes.txt" to \$fname
\begin{terminalcode}
$fname = 'notes.txt'
\end{terminalcode}
Assigning the value "vim" to \$editor
\begin{terminalcode}
$editor = 'vim'
\end{terminalcode}
"Touchs" the \$fname (in our case this is notes.txt) so that is created
if it doesn't exist already. \#\{\} tells to the compiler that \$fname is a variable.
\begin{terminalcode}
`touch #{$fname}`
\end{terminalcode}
.last returns the last element of \$\*. In \$* you find the arguments given by the user in the terminal (after note.rb). \verb+/\?\?$/+ is a regular expression. It means look a the end of the String (the end of the line is signed by \$) if there are two ?. There is a \verb$\$ in front of the question mark because it can mean other things in regular expressions. \verb$=~$ means as much as 'there is a regular expression coming. Please evaluate'

So what does this all mean now? Everytime when the last two characters of the last element of \$\* are ? then the following block is executed. Else the program jumps to 'elsif'.
\begin{terminalcode}
if $*.last =~ /\?\?$/
\end{terminalcode}
grep --ignore-case means using the bash command grep with the option 
--ignore-case. grep searches the named input for lines containing a match and prints the matching line. View "man grep" to learn  more about grep. The backticks \verb$``$ are used to tell the compiler that this is an extern command to be executed. We already know what \#\{\}, \#\{\$fname\} and \$\* are and do. .join('SEPERATING\_ELEMENT') returns all elements put together to one string. The different elements are seperated by SEPERATING\_ELEMENT in our case this is a whitespace. So we have at the end something like "element1 element2 element3". .chop does remove the last character of a string.

Overall this means that we are using the bash command "grep"  to look for a specific element(s) in our notes.txt file and assining the line(s), which consist this element(s), to the variable grep. In addition the result of \$\*.join(' ').chop.chop is assigned to the variable query.
\begin{terminalcode}
  grep = `grep --ignore-case "#{query = $*.join(' ').chop.chop}" #{$fname}`
\end{terminalcode}
.chomp does remove the newline at the end of a string. If there is one. .empty? does ask if there is something in a string. In this case it returns true. In the other case it returns false. "puts" does print the content of grep in the terminal. 
In other words these two lines mean: If grep is not empty. Then grep is printed in the terminal. Else the programm goes on with the block after else.
\begin{terminalcode}
  unless grep.chomp.empty? 
    puts grep
\end{terminalcode}
Nothing to tell you about the next two lines. You already know everything.
\begin{terminalcode}
  else 
    puts "Nothing known about \"#{query}\" :("
\end{terminalcode}
"end" tells the compiler that this is the end of unless. If the first element of \$\* equals to --edit, then the following block is executed.
\begin{terminalcode}
elsif $*.first == '--edit'
  `gnome-terminal --execute #{$editor} #{$fname}`
\end{terminalcode}
If the first element of \$\* equals to --hack, then the following block is
executed.
\begin{terminalcode}
elsif $*.first == '--hack'
  `gnome-terminal --execute #{$editor} \"#{__FILE__}\"`
\end{terminalcode}
If the first element of \$\* is not empty and it does not start with a "-"
then excute the following block.
\begin{terminalcode}
elsif not \$\*.empty? and not \verb+$*.first =~ /^-/+ then
\end{terminalcode}
Open a file and tell it that you want to (a)ppend something. Learn more about
Files: \verb$http://pleac.sourceforge.net/pleac_ruby/fileaccess.html$. The do command closes the file properly.
\begin{terminalcode}
  File.open($fname, 'a') do |file| 
  file.puts "#{Time.now.strftime("%Y/%m/%d\t%H:%M:%S")}\t#{$*.join(' ')}"
  end
\end{terminalcode}
If everything is falling appart (every if and elsif weren't excuted) then else 
and its block is called.
\begin{terminalcode}
else  
\end{terminalcode}
tail: prints the last 10 lines of a file.  See "man tail" for more information. The rest you already know.
\begin{terminalcode}
  puts `tail #{$fname}`
end
\end{terminalcode}

\end{document}

